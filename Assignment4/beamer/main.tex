\let\negmedspace\undefined
\let\negthickspace\undefined
\documentclass{beamer}
\usetheme{CambridgeUS}
\usepackage{csquotes}
\usepackage{comment}
\usepackage{enumerate}
\usepackage{amsmath,amssymb,amsthm}
\usepackage{graphicx}
\let\vec\mathbf
\newcommand{\myvec}[1]{\ensuremath{\begin{pmatrix}#1\end{pmatrix}}}
\providecommand{\brak}[1]{\ensuremath{\left(#1\right)}}
\providecommand{\pr}[1]{\ensuremath{\Pr\left(#1\right)}}


     \title{Assignment 4}
     \author{AKKASANI YAGNESH REDDY \\
     CS21BTECH11003 }
     \date{\today}
\logo{\large \LaTeX{}}
     
     \begin{document}
     \begin{frame}
     \maketitle    
     \end{frame}
     
     \logo{}
     
     \begin{frame}{Outline}
    \tableofcontents
     \end{frame}
     
     \section{Question}
     \begin{frame}{Question}
Show that
     \begin{align}
         \pr{A}=\pr{A|X\leq x}F(x)+\pr{A|X>x}[1-F(x)]
         \end{align}         
     \end{frame}
     
    \section{solution}
    \begin{frame}
        \textbf{Solution:}Lets define a random variable Y such that,
\begin{table}[h!]
    %\centering
    \begin{tabular}{|c|c|c|} \hline
    \textbf{Variable} & \textbf{Value} & \textbf{description} \\ \hline
    Y  & $1$ & If event A happens \\ \hline
    Y  & $0$ & If event A does not happen \\ \hline
    \end{tabular}
    \end{table}
    \end{frame}
    
    \begin{frame}
        Now lets take the given equation R.H.S.
\begin{align}
         &\pr{A|X\leq x}F(x)+\pr{A|X>x}[1-F(x)]\\
         \Rightarrow &\pr{Y=1|X\leq x}F(x)+\pr{Y=1|X>x}[1-F(x)]
 \end{align}
 \end{frame}
 \begin{frame}
By the definition of conditional probability.\\
\begin{align}
    \pr{Y=1|X\leq x}=\frac{\pr{(Y=1)(X\leq x)}}{\pr{X\leq x}} \\
    \pr{Y=1|X>x}=\frac{\pr{(Y=1)(X>x)}}{\pr{X>x}}
\end{align}
    \end{frame}
    
    \begin{frame}
    Also by the definition of cumulative probability,
\begin{align}
    &\pr{X \leq x}=F(x) 
    \end{align}
By the definition of probability,    
    \begin{align}
    &\pr{X \leq x}+\pr{X>x}=1 
    \end{align}
   Substituting $7$ in $6$ we get,
\begin{align}
    &F(x)+\pr{X>x}=1 \\
    \Rightarrow &\pr{X>x}=1-F(x) 
\end{align}
\end{frame}
\begin{frame}
    Substituting equations $4,5,6$ and $9$ in 3 we get,
\begin{align}
&\frac{\pr{(Y=1)(X\leq x)}}{F(x)}F(x)+ \frac{\pr{(Y=1)(X>x)}}{(1-F(x)}[1-F(x)] \\
\Rightarrow &\pr{(Y=1)(X\leq x)} + \pr{(Y=1)(X>x)}\\
\Rightarrow&\pr{Y=1}
\end{align}
\end{frame}
\begin{frame}
We showed that R.H.S is the same as L.H.S,
\begin{align}
    \pr{Y=1}=\pr{Y=1}
\end{align}
    \centering
       $\therefore$ \textbf{proved}.
\end{frame}         
     
     
     \end{document}
