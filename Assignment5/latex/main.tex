\let\negmedspace\undefined
\let\negthickspace\undefined
\documentclass[journal,12pt,twocolumn]{IEEEtran}

\usepackage{csquotes}
\usepackage{comment}
\usepackage{enumerate}
\usepackage{amsmath,amssymb,amsthm}
\usepackage{graphicx}
\let\vec\mathbf
\newcommand{\myvec}[1]{\ensuremath{\begin{pmatrix}#1\end{pmatrix}}}
\providecommand{\brak}[1]{\ensuremath{\left(#1\right)}}
\providecommand{\pr}[1]{\ensuremath{\Pr\left(#1\right)}}
 \title{Assignment 5}
     \author{AKKASANI YAGNESH REDDY \\
     CS21BTECH11003 \\}
     
     \begin{document}
     \maketitle
     \textbf{Question:}For a Poisson random variable $X$ with parameter $\lambda$ show that \\
     \begin{align}
    &(a)\pr{0<X<2\lambda} > \frac{(\lambda-1)}{\lambda}\\
    &(b)E[X(X-1)]=\lambda^{2}\\
    &(c)E[X(X-1)(X-2)]=\lambda^{3}
    \end{align}\\
     \textbf{Solution:}Given $X$ is a Poisson  random variable so,
     \begin{align}
\pr{X=x}&=\frac{e^{-\lambda}.\lambda^{x}}{x!}\\
         E(X)&=\lambda\\
         Var(X)&=\sigma_{X}^{2}=\lambda
     \end{align}
\textbf{(a)}From Chebyshev's inequality we know that,\\
for any $\epsilon>0$
\begin{align}
\pr{|X-\eta|\geq\epsilon}\leq\frac{\sigma_{X}^{2}}{\epsilon^{2}}
\end{align}
Also by definition of probability,
\begin{align}
    \pr{|X-\eta|\geq\epsilon}+\pr{|X-\eta|<\epsilon}=1
    \end{align}
    Substituting 8 in 7 the inequality becomes,
    \begin{align}
        \pr{|X-\eta|<\epsilon}>1-\frac{\sigma_{X}^{2}}{\epsilon^{2}}
    \end{align}
  Since this is Poisson distribution,
    \begin{align}
  \eta&=\lambda\\
  \sigma_{X}^{2}&=\lambda\\
  \epsilon&=\lambda
    \end{align}
    Substituting values in the equation.
    \begin{align}
    \pr{|X-\lambda|<\lambda}>1-\frac{\lambda}{\lambda^{2}}\\
    \Rightarrow\pr{|X-\lambda|<\lambda}>1-\frac{1}{\lambda}\\
        \Rightarrow\pr{0<X<2\lambda}>1-\frac{1}{\lambda}
    \end{align}
    \textbf{Hence proved.}\\\newpage
\textbf{(b)}To show,
\begin{align}
E[X(X-1)]=\lambda^{2}
\end{align}
Lets use the definitions,
\begin{align}
E[X]=\sum^{x=\infty}_{x=0}x\frac{e^{-\lambda}.\lambda^{x}}{x!}  
\end{align}
Also one the property of mean is that,
\begin{align}
    E[g(x)]=\sum^{x=\infty}_{x=0}g(x)\frac{e^{-\lambda}.\lambda^{x}}{x!}  
\end{align}
Here $g(x)=x(x-1)$ substituting it,
\begin{align}
    &E[X(X-1)]=\sum^{x=\infty}_{x=2}x(x-1)\frac{e^{-\lambda}.\lambda^{x}}{x!} \\
    \Rightarrow &E[X(X-1)]=e^{-\lambda}\lambda^{2}\sum^{x=\infty}_{x=2}\frac{.\lambda^{x-2}}{(x-2)!}
\end{align}
We know that by Taylors expansion,
\begin{align}
    e^{x}=\sum^{k=\infty}_{k=0}\frac{x^{k}}{k!}
\end{align}
Which is same as 
\begin{align}
    e^{\lambda}=\sum^{x=\infty}_{x=2}\frac{.\lambda^{x-2}}{(x-2)!}
\end{align}
Substituting it,
\begin{align}
    &E[X(X-1)]=e^{-\lambda}\lambda^{2}e^{-\lambda}\\
    \Rightarrow&E[X(X-1)]=\lambda^{2}
\end{align}
\textbf{Hence proved}
\\
\textbf{(c)}  using the above results we get
\begin{align}
&E[X(X-1)(X-2)]=\sum^{x=\infty}_{x=3}x(x-1)(x-2)\frac{e^{-\lambda}.\lambda^{x}}{x!}\\
 &E[X(X-1)(X-2)]=e^{-\lambda}\lambda^{3}\sum^{x=\infty}_{x=2}\frac{.\lambda^{x-3}}{(x-3)!}\\
 &E[X(X-1)(X-)2]=e^{-\lambda}\lambda^{3}e^{-\lambda}\\
 \Rightarrow&E[X(X-1)(X-2)]=\lambda^{3}
\end{align}
     
     \end{document}